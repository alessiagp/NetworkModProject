%%%%%%%%%%%%%%%%%%%%%%%%%%%%%%%%%%%%%%%%%
% Stylish Article
% LaTeX Template
% Version 2.2 (2020-10-22)
%
% This template has been downloaded from:
% http://www.LaTeXTemplates.com
%
% Original author:
% Mathias Legrand (legrand.mathias@gmail.com) 
% With extensive modifications by:
% Vel (vel@latextemplates.com)
%
% License:
% CC BY-NC-SA 3.0 (http://creativecommons.org/licenses/by-nc-sa/3.0/)
%
%%%%%%%%%%%%%%%%%%%%%%%%%%%%%%%%%%%%%%%%%

%----------------------------------------------------------------------------------------
%	PACKAGES AND OTHER DOCUMENT CONFIGURATIONS
%----------------------------------------------------------------------------------------

\documentclass[fleqn,10pt]{SelfArx} % Document font size and equations flushed left

\usepackage[english]{babel} % Specify a different language here - english by default
\usepackage{lipsum} % Required to insert dummy text. To be removed otherwise
\usepackage{csquotes}
\usepackage[makeroom]{cancel}
\usepackage[sorting=none, style=nature]{biblatex}
\usepackage{graphicx}
\usepackage{caption}
\usepackage{subcaption}
\usepackage{listings}
\lstset{language=Matlab}  
\bibliography{bibliography.bib}

\graphicspath{{./Figures/}}
%----------------------------------------------------------------------------------------
%	COLUMNS
%----------------------------------------------------------------------------------------

\setlength{\columnsep}{0.55cm} % Distance between the two columns of text
\setlength{\fboxrule}{0.75pt} % Width of the border around the abstract

%----------------------------------------------------------------------------------------
%	COLORS
%----------------------------------------------------------------------------------------

\definecolor{color1}{RGB}{39,104,13} % Color of the article title and sections
\definecolor{color2}{RGB}{222,121,32} % Color of the boxes behind the abstract and headings

%----------------------------------------------------------------------------------------
%	HYPERLINKS
%----------------------------------------------------------------------------------------

\usepackage{hyperref} % Required for hyperlinks

\hypersetup{hidelinks,
	colorlinks,
	breaklinks=true,
	urlcolor=color2,
	citecolor=color1,
	linkcolor=color1,
	bookmarksopen=false,
	pdftitle={Title},
	pdfauthor={Author},
	}

%----------------------------------------------------------------------------------------
%	ARTICLE INFORMATION
%----------------------------------------------------------------------------------------

\JournalInfo{Academic Year 2022-2023} % Journal information
\Archive{Network Modeling and Simulation [146100] - prof. L. Marchetti, F. Reali} % Additional notes (e.g. copyright, DOI, review/research article)

\PaperTitle{Validation of a Mathematical Model Describing the Dynamics of Chemotherapy for Chronic Lymphocytic Leukemia In Vivo} % Article title

\Authors{Alessia Guadagnin Pattaro\textsuperscript{1}, Giovanni Plazzotta\textsuperscript{1}, Annarita Zanon\textsuperscript{1}} % Authors
\affiliation{\textsuperscript{1}\textit{Master's degree in Quantitative and Computational Biology, University of Trento}} % Author affiliation

%\Keywords{Keyword1 --- Keyword2 --- Keyword3} % Keywords - if you don't want any simply remove all the text between the curly brackets
%\newcommand{\keywordname}{Keywords} % Defines the keywords heading name

%----------------------------------------------------------------------------------------
%	ABSTRACT
%----------------------------------------------------------------------------------------

\Abstract{This project is meant to analyse and replicate the results of a paper validating a mathematical model for different type of chemotherapeutic dosages for Chronic Lymphocytic Leukemia. The ODE-based model was recalculated up to the stability analysis of the stationary points and then implemented in MATLAB. 
The results of the simulations without treatment replicate exactly the ones reported in the main paper. When the treatment is present, the plots differ in a significative way, probably due to incorrect calculations or wrong formulations of the model. In addition to that, other criticalities are reported, for instance a wrong administration of the drug Cytarabine, and the twice accounting for the immune system.\\
A GitHub repository for the project was created \url{https://github.com/alessiagp/NetworkModProject} where the LaTeX scripts for the report, all the plots, and the MATLAB simulations scripts are stored.}

%----------------------------------------------------------------------------------------

\begin{document}

\maketitle % Output the title and abstract box

%\tableofcontents % Output the contents section

\thispagestyle{empty} % Removes page numbering from the first page

%----------------------------------------------------------------------------------------
%	ARTICLE CONTENTS
%----------------------------------------------------------------------------------------

\section{Introduction} % The \section*{} command stops section numbering

%\addcontentsline{toc}{section}{Introduction} % Adds this section to the table of contents

The paper under analysis sets itself the aim of building a mathematical model able to describe the interaction between malignant chronic lymphocytic leukemia (CLL) cells, effector cells, which in the case of cancer response are mainly NK cells (innate immunity) and T CD8 lymphocytes (adaptive immunity), and two different drugs, the Bruton tyrosine kinase (BTK) inhibitor Ibrutinib (Ibr) and an inhibitor of the enzyme topoisomerase, Cytarabine (Cyt), whose introduction has transformed CLL therapy and contributed to extend the overall survival of patients. Additional details about these small - molecule drugs will be presented in the sections below. While the formulation of a mathematical model describing the dynamics of cancer in an immunoactive environment is not innovative \textit{per se}, this paper sets itself apart in the sense that:
\begin{itemize}
\item It deals with a blood cancer instead of a solid tumor: in blood cancers, cell growth, survival dynamics and cellular interactions within the
tumor microenvironment likely differ considerably from solid tumors. It is expected that several parameters of previusly formulated ODE - based computational models need to be adapted for blood-borne cancers.
\item The majority of models use data generated by simulations to perform the parameter estimation, instead of experimental data. This paper, on the other hand, is strongly experimentally - oriented.
\item Even when data are derived from experiments, these are often performed \textit{in vitro} rather than \textit{in vivo}, and many studies have demonstrated conflicting results, in particular for what concerns optimal drug doses, between these two approaches. This paper, on the other hand, describes \textit{in vivo} experiments, performed on murine models injected with A20 CLL cells, aimed at the estimation of the cancer cells growth rate $r$ and of the cytotoxicity rates $\mu_{AC}$ as a function of therapy. \par
\vspace{0.4cm}
Chronic lymphocytic leukemia (CLL) is the most common type of blood cancer in adults in the Western world, with an incidence of 4.9 cases in 100000 people per year. It typically occurs in elderly patients, with the median age at diagnosis being 70 years. CLL is characterized by the clonal proliferation and accumulation of mature B lymphocytes in secondary lymphoid organs, spleen, peripheal blood, and bone marrow. \cite{cll-burger-med, cll-rozman-med}  There is no known cause for this disease, but it is suspected to have a genetic component. Loss or addition of large chromosomal material followed by additional mutations, that render the leukemia increasingly aggressive, are often observed. Additionally, mutations in \textit{IGHV} (immunoglobulin heavy variable) genes distinguishing different types of clinical behaviours of CLL and are prognostic of patent outcome \cite{immunogl-med}. \par
%Treatment of Chronic Lymphocytic Leukemia - Burger Jan A.
%Chronic Lymphocytic Leukemia - Rozman et al
%Immunoglobulin heavy variable (IGHV) genes and alleles - Xochelli et al 
\vspace{0.4cm}
Different treatment strategies are available for patients suffering from CLL. It is important to notice that studies on early-stage disease were unable to show a benefit of early therapeutic interventions: treatment of patients with early stage CLL did not result in a survival benefit. For this reason, and not to fruitlessly trigger the development of drug resistance, patient in these stages should not be treated, but only monitored, until the disease becomes \textit{active}. The degree of \textit{activity} of CLL can be assessed using the following guidelines:
\begin{itemize}
    \item Progressive lymphocytosis with an increase of $\geq 50 \%$ over a 2 month period, or lymphocyte doubling time of less than 6 months.
    \item Worsening of anemia and/or thrombocytopenia
    \item Massive nodes (i.e., $\geq 10$ cm in longest diameter)
    \item Massive or symptomatic splenomegaly and hepatomegaly
    \item Autoimmune complications
    \item Functional extranodal involvement (e.g., skin, kidney, lung, spine)
    \item Significant weight loss and fatigue, fevers above 38 degrees for more than two weeks, night sweats. 
\end{itemize}
When the treatment becomes necessary, specialists can choose between different classes of drugs, targeting different aspects of the cellular structure (e.g. surface antibodies), metabolism and external microenvironment:
\begin{itemize}
\item Cytostatic agents: having the goal to stop cellular proliferation by interfering with the replication process. Examples are purine analogs, such as Cytarabine (Cyt), one of the two drugs that we will discuss in this presentation.
\item Monoclonal Antibodies: specifically built to interact with surface antigens that have been documented to be overexpressed in malignancies, like CD-20 and CD-52 receptors in B-cells blood cancers. They have the goal of guiding the immune system towards the cancerous cells. Examples are Rituximab and Alemtuzumab.
\item Signalling - targeting agents: having the goal of interfering with the embedded signalling pathways in order to trigger apoptosis. Ibrutinib (Ibr), the second drug discussed in this presentation, is a Bruton Tyrosine Kinase inhibitor that hinders the downstream propagation of the signal generated by bounded BCR receptors, stopping the activation of B cells survival pathways. 
\item Immunotherapy: different CART approaches are being explored.




%------------------------------------------------

\section{Chemotherapeutic drugs}
\subsection{Chemotherapeutic drugs}
\subsubsection{Ibrutinib (Ibr)}
Ibrutinib (Ibr) is a chemotherapic drug (Fig. \ref{fig:Ibr}) used as a deactivator of Bruton tyrosine kinase (BTK), that acts as molecular mediator of pathogenesis and is essential for B cell receptor signaling\cite{ibr-1}.  
Whenever B cells receptor signaling has an aberrhant behaviour alongside antigen-dependent activation, is involved in the pathogenesis of many lymphocytes-related malignancies.\\
Ibr blocks B cell antigen receptor signalling through an irreversible covalent bond with Cys-481 of BTK, hence reducing malignant B cell proliferation and inducing cell death.\\
The drug reaches its maximum concentration in plasma ($953\,\text{ng}\cdot\text{h}/\text{mL}$ at dosage of $560\,\text{mg}/\text{day}$) in 1-2 h and is widely distributed in the body. The major route of elimination is metabolism. It is metabolised by hepatic cytochrome P450 3A enzymes. It has an elimination half-life of 4-6 h via faeces.\\
\begin{figure}[htbp!]
	\centering
	\includegraphics[scale=0.1]{Ibrutinib.png}
	\caption{Molecular structure of chemotherapeutic drug Ibrutinib}
	\label{fig:Ibr}
\end{figure}

%--------------------------
\subsubsection{Cytarabine (Cyt)}
Cytarabine (Cyt) is a medication used in treatment of leukemias and lymphomas. As it can be seen in Fig. \ref{fig:Cyt}, it is a nucleoside (pyrimidine analog) with arabinose sugar, also called arabinosylcytosine, and is an antimetabolite and antineoplastic.
The sugar moiety induces the rotation of Cyr within the DNA, blocking DNA replication during the S-phase of cellular replication. It also acts on DNA polymerase and its maximum effects are seen after the time equivalent to a full cell cycle (8-12 h).
Cyt has two types of metabolites: 
\begin{itemize}
	\item inactive metabolites, from deamination as soon as the drug enters the plasma
	\item active metabolites (Cyt triphosphate, CytTP), after being transported into the cell, and after phosphorylation.
\end{itemize}
The active metabolite competitively inhibits DNA polymerase, it is incorporated into DNA where it acts as chain terminator, leading to incomplete DNA and cell death.\\
CytTP has a saturation level, leading to accumulation of the metabolite in cells, a lower drug selectivity of cancer cells, and a higher degree of myelosuppression.\cite{cyt-1, cyt-2}
\begin{figure}[htbp!]
	\centering
	\includegraphics[scale=0.1]{Cytarabin.png}
	\caption{Molecular structure of chemotherapeutic drug Cytarabine}
	\label{fig:Cyt}
\end{figure}
\section{Test}
\noindent Ciaone



%------------------------------------------------

\section{Paper analysis}
\subsection{Mathematical model}
Based on previous studies, an ODE model was formulated to explain the interaction between CLL cells, immune cells and chemotherapeutic drugs:

\[
\begin{cases} 
	\frac{dA}{dt} = rA \bigl( 1 - \frac{A}{K} \bigr) - \mu_A AE - \frac{\mu_{AC} AC}{a+C} & (1)\\ \\
	\frac{dE}{dt} = -\mu_E E + \frac{pAE}{c+A} - \mu_{AE} AE - \frac{\mu_{EC} EC}{b+C} & (2) \\ \\
	\frac{dC}{dt} = \sum_{m=0}^{N-1} d\delta (t-m\tau) - \mu_{C} - \frac{\mu_{CA} CA}{a+A} & (3) 
\end{cases}
\]

$\frac{dA}{dt}$ describes the dynamic of A20 mCherry cells. The first term reflects the assumption that cancer cells follow a logistic growth with \textit{instantaneous growth rate} $r$ and carrying capacity $K$. The carrying capacity represents the maximal tumor cell number that the system is able to host. A logistic growth is a reasonable assumption for cancer growth, since it takes place in a competitive environment with limited resources. Cancer cells can be killed by both NK cells (innate immunity) and T CD8 lymphocytes (adaptive immunity): these were considered together in the single variable $E$, whose dynamics is described in the second equation. The overall killing activity of immune cells can be modelled with the law of mass action, assuming a \textit{killing efficiency} $\mu_{A}$. The last term represents the effect of the treatment on tumor cells: the numeRator simulates the interaction between tumor cells and drug molecules with the law of mass action, with a \textit{killing efficiency} $\mu_{AC}$, while the denominator introduces a Michaelis Menten drug saturation response, for which the whole term converges to the maximum killing rate $\mu_{AC} \cdot A$ as the drug concentration $C$ is brought to infinity. This is reasonable strategy to model the drug response, since we expect a plateau in the effectiveness of the drug as its concentration is increased. In this last term, $a$ represents the drug concentration needed to reach half of the maximum killing rate. \par
\vspace{0.4cm}
$\frac{dE}{dt}$ describes the dynamic of immune effector cells. Their number is assumed to decline with rate $\mu_{E}$ due to natrual death. It is known from literature that cancer cells are able to induce a recruitment effect on immune cells, due to the pro inflammatory environment defined by cancer itself. This is represented by the second term. The recruitment effect increases as the tumor mass grows, but up to a certain maximum rate, represented by $p \cdot E$. Additionally, $c$ is the number of cancer cells by which the immune system response is half of its maximum. It is also known from literature that T CD8 and NK cells undergo apoptosis after a certain number of encounters with malignant cells: the cytotoxic molecules released against cancer inevitably cause damage to immune cells too, and this is modelled by the third term, using again the law of mass action. Finally, the drug administered to treat CLL also kills host immune cells: this is modelled as described above for cancer cells. \par
\vspace{0.4cm}
$\frac{dC}{dt}$ describes the first-order pharmacokinetics of a drug with an external source, with $C$ being the concentration of the drug in the bloodstream. A dose $d$ of the drug is injected every $\tau$ hours. By modeling the injection as a shifted Dirac Delta function $\delta (t - m\tau)$, the $m^{th}$ dose raises $C(t)$ by $d$ units at $t=m\tau$. It was assumed that the drug was eliminated from the body with a rate $\mu_C$, calculated as $\mu_C = \frac{\ln 2}{t_{1/2}}$, where $t_{1/2}$ is the elimination half-life of the drug (1–3 h for Cyt and 4–6 h for Ibr). The drug concentration can also be depleted by the interaction with cancer cells, having rate $\mu_{CA}$. Finally, $a$ represents the drug concentration producing $50\%$ of the maximum activity in the A20 mCherry cell population. 


\subsection{Stability Analysis}
\subsubsection{Stability when $d=0$, \textit{i.e. without treatment}}
The (1)-(3) model depends on time in the chemotherapy levels. Since equilibra cannot explicitly depend on $t$ for each $t\in \mathbf{R}$, we can approximate $\sum_{m=0}^{N-1} d\delta(t - m\tau)$ using a uniform drug injection that takes the form $\frac{d}{\tau}$. Only non-negative equilibria are considered and all initial conditions are assumed to be positive

\[
	\begin{cases}
	0 = rA \bigl( 1 - \frac{A}{K} \bigr) - \mu_A AE - \frac{\mu_{AC} AC}{a+C} & (A1)\\ \\ 
	0 = -\mu_E E + \frac{pAE}{c+A} - \mu_{AE} AE - \frac{\mu_{EC} EC}{b+C} & (A2) \\ \\
	0 = \frac{d}{\tau} - \mu_CC - \frac{\mu_{CA} CA}{a+A} & (A3) 
	\end{cases} 
\]

Wihout treatment ($d=0$) and if $C^* \neq 0$, we find, from eq (A3)

\[
	\frac{d}{\tau} - \mu_C C^* - \frac{\mu_{AC}C^* A}{a+A} = 0 \]
\[- \mu_C C^* = \frac{\mu_{AC}C^* A}{a+A} \rightarrow -\mu_C = \frac{\mu_{CA} A}{a+A} \rightarrow\]
\[\rightarrow -\mu_Ca - \mu_CA = \mu_{CA} A \rightarrow A^* = -\frac{\mu_{CA}}{\mu_C + \mu_{CA}} < 0\]
Since we have assumed all parameters to be positive. Then, in there is no treatment $C^*=0$, which is pretty obvious, since $C$ is the amount of chemotherapic drug.\\
Under this condition, (A1) becomes
\[ rA \left( r - \frac{A}{K} \right) - \mu_A EA = 0 \rightarrow A \left( r - \frac{rA}{K} - \mu_A E \right) = 0 \]
Which is certainly true for $A_0^*=0$. Under this conditon, we find from (A2) that $-\mu_E E = 0$ which means $E_0^*=0$.\\
We find another fixed point by solving
\[ r - \frac{rA^*}{K} - \mu_A E* = 0 \rightarrow A^* = \frac{rK-\mu_AE^*K}{r} \]
in the condition of no treatment, (A2) reduces to

\[ E\bigl( -\mu_E + \frac{pA}{c+A} - \mu_{EA} A \bigr) = 0 \]

which is true for $E_1^*=0$. Substituting it into $A^*$ we get $A_1^* = \frac{rK - \mu_A 0 K}{r} = K$.\\
Other fixed points can be found by solving
\[ -\mu_E + \frac{pA^*}{c+A^*} - \mu_{EA} A^* = 0 \rightarrow {A^*}^2 \mu_{EA} + A^* (\mu_E + \mu_{EA}c - p) + \mu_Ec = 0 \]
By substituting the values from Table 1 we get (for Cyt)

\[ {(A^*)}^2 4\times 10^{-15} + A^* (4\times 10^{-5} + 4\times 10^{-15} \cdot 10^2 - 4\times 10^{-14}) + \]
\[+ 4\times 10^{-5} \cdot 10^2 = 0 \rightarrow A^*_{2,3} < 0 \]
which gives $A^*_2 \approx -100$ and $A^*_3 \approx -10^{-10}$ that have no meaning, biologically speaking. Thus, for $d=0$ we have only two equilibria: $Eq_0^* = \{ A^*=0, E^*=0, C^*=0\}$ and $Eq_1^* = \{ A^*=K, E^*=0, C^*=0\}$.\\

To study the stability of the equilibria of system (A1)-(A3), we need to compute the eigenvalues $\lambda = [\lambda_1,\lambda_2,\lambda_3]$ of the Jacobian matrix $J$
\[ \mathbf{J} = \begin{bmatrix} \frac{\partial A1}{\partial A} & \frac{\partial A1}{\partial E} & \frac{\partial A1}{\partial C} \\ \\
\frac{\partial A2}{\partial A} & \frac{\partial A2}{\partial E} & \frac{\partial A2}{\partial C}\\ \\
\frac{\partial A3}{\partial A} & \frac{\partial A3}{\partial E} & \frac{\partial A3}{\partial C} \end{bmatrix} \]

For $Eq_0^*$ we obtain

\[ J = \begin{pmatrix} r&0&0 \\ 0&-\mu_E&0 \\ 0&0&-\mu_C \end{pmatrix} \]

that has eigenvalues $\lambda = [0.01,-4\times 10^{-5},-0.231]$ for Cyt and $\lambda = [0.01,-4\times 10^{-5},-0.116]$. For the equilibrium to be stable, all the (real parts) of the eigenvalues must be negative. So $Eq_0^*$ is NOT asymptotically stable.\\

For $Eq_1^*$ we have

\[ J = \begin{pmatrix} -r & -\mu_{AE}K & -\frac{\mu_{AC}K}{a} \\ 0 & -\mu_E + \frac{pK}{c+K} - \mu_{EA}K \\ 0&0& -\mu_C - \frac{\mu_{CA}K}{a+K} \end{pmatrix} \]

which has eigenvalues $\lambda = [-0.01, -4\times 10^{-5}, -0.35]$. So $Eq_1^*$ is asymptotically stable.
\subsubsection{Stability of periodic tumour-free solutions} 
Eq. (A1) admits the solution $A^* = E^* = 0$ with chemotherapy dynamics satisfying

\[ \frac{dC}{dt} = \sum_{m=0}^\infty \mathbf{d} \delta (t-m\tau) -\mu_C C \]

The delta function allows the determine the dynamics between consecutive injections of the drug, \textit{i.e.} when the dirac delta vanishes
\[ \frac{dC}{dt} = -\mu_C C, \quad \text{when} \ n\tau \leq t < (n+1)\tau  \]

If we set the initial condition $C(0) = d$

\[ C(t) = C(0) e^{-\mu_C t} = d e^{-\mu_C t}, \quad \text{when} \quad 0 \leq t < \tau \]
\[C(t) = d e^{-\mu_C t} + \tau e^{-\mu_C \tau}, \quad \text{when} \quad \tau \leq t < 2\tau \] 
\[\dots\]
\[ C(t) = d e^{-\mu_C t} + n\tau e^{-\mu_C \tau}, \quad \text{when} \quad n\tau \leq t < (n+1)\tau \]

In the limit of large $n$, $C(t)$ converges to a periodic cycle $\tilde{C}(t)$. The associated cancer-free state will be

\[ \begin{cases}
	A^* = E^* = 0 \\ 
	\tilde{C}(t) =  d e^{-\mu_C t} + n\tau e^{-\mu_C \tau} 
\end{cases} \]
\subsubsection{Linearization around equilibria} 
This method allows the study of local properties of a non-linear system of differential equations. Let $Eq = (A^*,E^*,C^*)$ be an equilibrium of (1)-(3). By defining $x_1 = A(t) - A^*, x_2 = E(t) - E^*$ and $x_3 = C(t) - C^*$ and then 


\subsection{Parameter estimation}
Two parameters of the model, the \textbf{\textit{in vivo} growth rate $r$ of A20 mCherry cells} and the \textbf{cytotoxicity rate $\mu_{AC}$} in the presence of the drug were experimentally derived with an \textit{in vivo} approach, for which 20 murine models were considered. \par
\vspace{0.4cm}
To measure the instantaneous growth rate, 20 mice were inoculated with A20 murine leukemic cells. On day 16 after inoculation, blood was collected from the tail veins of four randomly chosen mice, and again on day 22 from four other mice. The proportion of A20 cells over the total was estimated using flow cytometry. Assuming a logistic growth (appropriate for cancer growth, since it takes place in a competitive environment with limited resources), we can compute the instantaneous growth rate as: 
\[ r = \frac{\ln{N(t)/N(0)}}{t} = \frac{\ln{16338/3662}}{144} = 0.01\ h^{-1} \]
Where $N(0), N(t)$ are the number of cells at times $0$ (16 days after inoculation) and $t$ (22 days, and so 144 hours, after inoculation).\\ \par
\vspace{0.4cm}
The second parameter, $\mu_{AC}$, is crucial for the model since it represents the efficiency with which a drug can kill cancer cells. Researchers were interested in computing $\mu_{AC}$, by the means of \textit{in vivo} experiments on murine model bearing the A20 cells, for the effect of two different drugs, Cytarabine (Cyt) and Ibrutinib (Ibr), in different doses. To apply the desired protocols, developed after reviewing the literature in which Cyt and Ibr had been used \textit{in vivo}, researchers divided the 20 mice into 5 groups:\begin{enumerate}
	\item Control group, which only received PBS.
	\item Cyt Low group, which received $0.12$ mg/kg of Cyt for 5 days. This is equivalent to injecting $5.94 \cdot 10^{15}$ molecules of Cyt at each
	administration. 
	\item Cyt High group, which received $62.5$ mg/kg of Cyt for 3 days. This is equivalent to injecting $3 \cdot 10^{18}$ molecules of Cyt at each
	administration. 
	\item Ibr Low group, which received $9$ mg/kg of Ibr on days 1-5 and 8-10. This is equivalent to injecting $2.5 \cdot 10^{17}$ molecules of Ibr at 
	each administration. 
	\item Ibr High group, which received $18$ mg/kg of Ibr on days 1–5 and 8–10. This is equivalent to injecting $5 \cdot 10^{17}$ molecules of Ibr at 
	each administration. 
\end{enumerate}
To estimate $\mu_{AC}$ in groups 2 to 5, blood was collected from all mice, on the day of the initiation of the treatment and 2 days after the last treatment. At each of the two time points and in each treated mouse, the per cent change in the frequency of A20 cells relative to the average
frequency in the control group was calculated. Then, for each treated group, these individual measurements were then averaged. This led to the estimation of the \textit{experimental cell growth inhibition percentage} for the 4 treated groups (Fig. \ref{fig:tables}). \par
\begin{figure} [htbp!]
\centering
\begin{subfigure}{0.49\textwidth}
\centering
\includegraphics[scale = 0.18] {conc.png}
\end{subfigure}
\begin{subfigure}{0.49\textwidth}
\centering
\includegraphics[scale = 0.18] {inhibition.png}
\end{subfigure}
\caption{Drug dosages and resulting experimental cell growth inhibition percentage in the four treatment groups}
\label{fig:tables}
\end{figure}
At this point, a total of 12 deterministic simulations were performed (Fig. \ref{fig:12sims}), 3 for each treatment group, using $(A = 5.4 \cdot 10^{4}, E = 2500, C = 0)$ as starting point. In the context of the same treatment group, the simulations set themselves apart for the numerical choice of $\mu_{AC}$: Cyt Low was tested with $\mu_{Ac} = 0.0001$, $\mu_{Ac} = 0.001$ and $\mu_{Ac} = 0.003$. Cyt High was tested with $\mu_{Ac} = 0.005$, $\mu_{Ac} = 0.012$, $\mu_{Ac} = 0.02$. Ibr Low was tested with $\mu_{Ac} = 0.001$, $\mu_{Ac} = 0.0041$, $\mu_{Ac} = 0.005$. Ibr High with $\mu_{Ac} = 0.002$, $\mu_{Ac} = 0.0042$, $\mu_{Ac} = 0.006$. Black lines represent cancer evolution without treatment. The goal was to find, for each treatment protocol, the value for $\mu_{Ac}$, among the one proposed, that gives a \textit{simulated cell growth inhibition percentage} similar to the one obtained from the experiments (Fig. \ref{fig:finaltab}). 

\begin{figure} [htbp!]
    \centering
    \includegraphics[scale = 0.20] {4.png}
    \caption{Evolution of the system in the absence of treatment (black lines) and with different choices for $\mu_{Ac}$, in the context of every treatment protocol}
    \label{fig:12sims}
\end{figure}

\begin{itemize}
    \item Cyt Low has a target cell growth inhibition percentage of $9 \%$. This is best reached by using $\mu_{Ac} = 0.001$, which provided $10 \%$ of growth inhibition. 
    \item Cyt High has a target cell growth inhibition percentage of $58 \%$. This is best reached by using $\mu_{Ac} = 0.012$, which provided $59 \%$ of growth inhibition. 
    \item Ibr Low has a target cell growth inhibition percentage of $43.5 \%$. This is best reached by using $\mu_{Ac} = 0.0041$, which provided $43.4 \%$ of growth inhibition. 
    \item Ibr High has a target cell growth inhibition percentage of $44.5 \%$. This is best reached by using $\mu_{Ac} = 0.0042$, which provided $44 \%$ of growth inhibition. 
\end{itemize}
\begin{figure} [htbp!]
    \centering
    \includegraphics[scale = 0.28] {table.png}
    \caption{Comparision between the experimental cell growth inhibition percentages and the simulated ones, obtained by fine-tuning $\mu_{Ac}$ for each protocol}
    \label{fig:finaltab}
\end{figure}


\subsection{Simulation of a combined therapy}
The last simulation performed by the authors had the goal of estimating the \textit{simulated cell growth inhibition percentage} of a treatment comprising both Cytarabine (Cyt) and Ibrutinib (Ibr). The authors were able to simulate a combined therapy without modifying the system of equations (which, in principle, is built to model the injections of a single type of drug) by modyfing the parameters of the model so that the combined therapy could be simulated as a single treatment (Fig. \ref{fig:combo}). In particular, the mixed treatment was obtained by combining Cyt High ($62.5$ mg/Kg of Cyt, $3 \cdot 10^{18}$ molecules of Cyt at each administration) and Ibr Low ($9$ mg/Kg of Ibr, $2.5 \cdot 10^{17}$ molecules of Ibr at each administration) for 8 days of treatment (5 days of treatment, 2 days of break, and another 3 days of treatment). This means injecting a total of $3.25 \cdot 10^{18}$ molecules at each administration. Since under these conditions Cyt represents approximately $92 \%$ of the total molecules, the combined treatment was parameterized by setting $\mu_{C} = (0.231 \cdot 0.92 + 0.116 \cdot 0.08) = 0.221$ and $\mu_{AC} = 0.012 + 0.0041 = 0.0161$. The simulation run with these parameters predicted a $95 \%$ cell growth inhibition. \par
\begin{figure}[htbp!]
    \centering
    \includegraphics [scale = 0.27] {combined.png}
    \caption{Evolution of the system in the absence of treatment (black line) and with a combined protocol (red line)}
    \label{fig:combo}
\end{figure}


%------------------------------------------------

\section{MATLAB simulations}
\subsection{Implementation}
The deterministic simulations presented in the paper were replicated by using the MATLAB suite \texttt{ode45}. This library, dedicated to solve systems of differential equations in the form $y'=f(t,y)$, implemnts an adaptive-step Runge-Kutta integration algorithm of the fourth-order. 
We developed the following script to solve the system of equation (1), (2), (3) for each treatment condition.

\begin{lstlisting}

Tf = 1440; % Final time
tspan = [0;Tf];
X0 = [500,2500,0]; % Starting conditions [A(0),E(0),C(0)]
op1 = odeset('MaxStep',1,'RelTol',1e-2,'AbsTol',1e-4);

[t,X] = ode45(@model,tspan,X0,op1);

function dX = model(t,X)

%%% Parameters definition
r = 0.01;   % A20 growth rate [h^-1]
K = 4e6;    % Max A20 number [cells/mouse]
mu_ac = 0.012; % Cyt 62.5 mg/kg for 3 days cytotoxicity rate [h^−1]
mu_ca = mu_ac * 10; % Deactivation of drug due to killing of A20 [h^-1]
d = 1.25 * 2.4e18; % Cyt 62.5 mg/kg dose [molecules/mouse]
mu_c = 0.231; % Cyt chemical deactivation rate [h^-1]
mu_a = 2e-12; % Effectors-A20 interaction coefficient [h^-1]
a = 2e3;    % Drug amount producing 50% of max effect on A20 [molecules]
p = 4e-14; % Production rate of effectors stimulated by A20 [h^-1]
mu_ec = 417; % Mortality rate of drug on effector cells [h^-1]
c = 1e2;    % Num of A20 producing 50% of max immune activation [cells]
b = 5e6;    % Drug amount producing 50% max effect on healty cells [molecules]
mu_ea = 4e-15; % A20-effectors interaction coefficient [h^-1]
mu_e = 4e-5; % Death rate of effecors [h^-1]
tau = 24;

A = r*X(1)*(1-(X(1)/K)) - mu_a*X(1)*X(2) - ((mu_ac*X(1)*X(3))/(a+X(3)));
E = -mu_e*X(2) + (p*X(1)*X(2))/(c+X(1)) - mu_ea*X(1)*X(2)  - (mu_ec*X(2)*X(3))/(b+X(3));
if t(1) > 336 && rem(round(t(1),0),tau) == 0 && t(1) < 384
    C = d - mu_c*X(3) - (mu_ca*X(3)*X(1))/(a+X(1)); 
else
    C = - mu_c*X(3) - (mu_ca*X(3)*X(1))/(a+X(1));
end
dX = [A;E;C];
end

\end{lstlisting}


\subsection{Results}
The \textbf{growth rate \textit{in vivo} $r$ of A20 mCherry cells} was calculated according to

\[ r = \frac{\ln{N(t)/N(0)}}{t} = \frac{\ln{16338/3662}}{144} = 0.01\ h^{-1} \]

Where $N(0), N(t)$ are the number of cells at times $0$ (16 days after inoculation) and $t$ (22 days after inoculation).\\
To determine the \textbf{efficacy of Cyt and Ibr \textit{in vivo}}, five groups of mice were treated with different doses and time periods. Tail vein blood was collected on the day of initiation of treatment and two days after the last treatment. 
At each time point, the percent change in frequency of A20 cells in each treated mouse relative to the average frequency in the Control group was calculated. From these data, the average A20 frequency change for each treated group was obtained. 
The difference between the average frequency in each test group and the Control group represents the percent growth inhibition as a function of treatment. Growth inhibition due to Cyt was dose dependent (low dose 9\%, high dose 58\%), whereas inhibition due to Ibr was not (low and high doses about 44\%).
Based on previous studies and on the \textit{in vivo} experiments, an ODE model was formulated to explain the interaction between CLL cells and chemotherapeutic drugs

\[
\begin{cases} 
	\frac{dA}{dt} = rA \bigl( 1 - \frac{A}{K} \bigr) - \mu_A AE - \frac{\mu_{AC} AC}{a+C} & (1)\\ \\
	\frac{dE}{dt} = -\mu_E E + \frac{pAE}{c+A} - \mu_{AE} AE - \frac{\mu_{EC} EC}{b+C} & (2) \\ \\
	\frac{dC}{dt} = \sum_{m=0}^{N-1} d\delta (t-m\tau) - \mu_{C} - \frac{\mu_{CA} CA}{a+A} & (3) 
\end{cases}
\]

$\frac{dA}{dt}$ describes the dynamic of A20 mCherry cells. The positive term corresponds to the logistic cancer growth characterized by the coefficient $r$ which is limited by the maximal tumor cell number $K$. 
The negative terms correspond to A20 cells being killed by effector cells with rate $\mu_A$ and to the log-kill hypothesis, with a Michaelis–Menten drug saturation response $a + C$. $\mu_{AC}$ is the death rate resulting from the action of the drug on cancer cells.
$\frac{dE}{dt}$ describes the dynamic of immune effector cells. $\mu_{E}$ is the natural death rate of effector cells. $p$ is the production rate of effector cells stimulated by cancer cells. $c$ is the number of cancer cells by which the immune system response is half of its maximum. 
$\mu_{EA}$ is the interaction coefficient between cancer and effector cells affecting immune populations. $\mu_{EC}$ is the mortality rate due to the action of chemotherapy on effector cells and $b$ is the drug amount for which such effect is half of its maximum.
$\frac{dC}{dt}$ describes the first-order pharmacokinetics of a drug with an external source. A dose $d$ of the drug is injected every $\tau$ hours. By modeling the injection as a shifted Dirac Delta function $\delta (t - m\tau)$, the $m^{th}$ dose raises $C(t)$ by $d$ units at $t=m\tau$. $\mu_C$ is the deactivation rate calculated as $\mu_C = \frac{\ln 2}{t_{1/2}}$, where $t_{1/2}$ is the elimination half-life (1–3 h for Cyt and 4–6 h for Ibr). $\mu_{CA}$ is the rate at which drug molecules attack cancer cells. 
$a$ represents the drug concentration producing 50\% of the maximum activity in the A20 mCherry cell population. A mathematical analysis of the model (1)–(3) found that the system is characterized by three fixed points, one of which is asymptotically stable.
For the \textbf{parameters} to have biological meaning, they must be positive. The initial conditions at $t=0$ are the initial number of A20 cells $A(0) = 5 \times 10^4$ cells/mouse, the initial number of effector cells $E(0) = 2500$ cells/mouse and no drug before the treatment $C(0)=0$. The number of drug molecules is calculated using $\frac{mN_a}{M}$, where $m$ is the mass of the drug, $N_a$ is Avogadro's number and $M$ the molar mass of the drug.
A \textbf{sensitivity analysis} determined the range of parameter $\mu_{AC}$ which has the most significant impact on cancer cells under the different treatments. To test the growth inhibitory effects of Cyt combined with Ibr, a simulation was ran that predicted 95\% of cell growth inhibition, which represents an 85\% increase in killing efficiency compared to separate treatment.
%prova per vedere se mi fa i commit giusti


%------------------------------------------------
\phantomsection
\section{Conclusions} % The \section*{} command stops section numbering
Cyt is a drug not currently used to treat CLL that is more cytotoxic to A20 cells \textit{in vitro} than Ibr, which suggests a repurposing of this cancer drug. Indeed, this prediction was vindicated when tested in the animal model presented in this paper. 
A numerical simulation of the potential effect of Cyt plus Ibr on A20 cells predicted that such a combination could increase cytotoxicity and inhibit cancer cell growth by up to 95\%. It would now be valuable to test this combined treatment \textit{in vivo}, especially as these drugs have different modes of action. 
This model exhibits several stable states that depend on biologically related parameters: Analysis of stability shows that the free-tumor $Eqm^*_0$ equilibrium is not stable, which means that if there are no more cancer cells and the treatment is stopped, the model is in equilibrium without growth, albeit unstable. This may also represent a state of cancer cell dormancy, an adaptive strategy used by cancer cells to overcome drug cytotoxicity. This stage may persist until complete metabolism of the drug, which would allow tumor growth to recur. 
The fixed point $Eqm^*_1$ is a stable equilibrium reached when the number of cancer cells reaches its maximum. The system is not stable at $Eqm^*_2$ equilibrium with periodical chemotherapy, which is obtained when treatment is stopped before the cancer cells are completely removed.
The step of calculating the growth rate of cancer cells \textit{in vivo} has to be carried out for each type of cancer cell, and it's easier to perform for blood-borne cancers. Current models do not easily help personalized chemotherapy dosing, partially because tumor cell growth rates vary between patients. The validation of experimental model with simulations studies can aid in selecting an optimal range of dosages to test.
Furthermore, the model can be used to simulate combination drug therapy: the model predicted that a combination of Cyt and Ibr would lead to about 95\% killing of A20 cells. Such high rates of killing are not expected in clinical practice, mainly due to subsequent toxicities. This model allows to predict a potentially effective new combination of drugs. Further experiments in vivo may reveal that two drugs with different modes of action may have acceptable efficacy at a lower dosage. 

%\addcontentsline{toc}{section}{Conclusions} % Adds this section to the table of contents

%----------------------------------------------------------------------------------------
%	REFERENCE LIST
%----------------------------------------------------------------------------------------
%\newpage
%\phantomsection
\printbibliography
%\bibliographystyle{unsrt}
%\bibliography{bibliography.bib}

%----------------------------------------------------------------------------------------

\end{document}
