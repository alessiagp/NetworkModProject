The paper under analysis sets itself the aim of building a mathematical model able to describe the interaction between malignant chronic lymphocytic leukemia (CLL) cells, effector cells, which in the case of cancer response are mainly NK cells (innate immunity) and T CD8 lymphocytes (adaptive immunity), and two different drugs, the Bruton tyrosine kinase (BTK) inhibitor Ibrutinib (Ibr) and an inhibitor of the enzyme topoisomerase, Cytarabine (Cyt), whose introduction has transformed CLL therapy and contributed to extend the overall survival of patients. Additional details about these small - molecule drugs will be presented in the sections below. While the formulation of a mathematical model describing the dynamics of cancer in an immunoactive environment is not innovative \textit{per se}, this paper sets itself apart in the sense that:
\begin{itemize}
\item It deals with a blood cancer instead of a solid tumor: in blood cancers, cell growth, survival dynamics and cellular interactions within the
tumor microenvironment likely differ considerably from solid tumors. It is expected that several parameters of previusly formulated ODE - based computational models need to be adapted for blood-borne cancers.
\item The majority of models use data generated by simulations to perform the parameter estimation, instead of experimental data. This paper, on the other hand, is strongly experimentally - oriented.
\item Even when data are derived from experiments, these are often performed \textit{in vitro} rather than \textit{in vivo}, and many studies have demonstrated conflicting results, in particular for what concerns optimal drug doses, between these two approaches. This paper, on the other hand, describes \textit{in vivo} experiments, performed on murine models injected with A20 CLL cells, aimed at the estimation of the cancer cells growth rate $r$ and of the cytotoxicity rates $\mu_{AC}$ as a function of therapy. \par
\vspace{0.4cm}
Chronic lymphocytic leukemia (CLL) is the most common type of blood cancer in adults in the Western world, with an incidence of 4.9 cases in 100000 people per year. It typically occurs in elderly patients, with the median age at diagnosis being 70 years. CLL is characterized by the clonal proliferation and accumulation of mature B lymphocytes in secondary lymphoid organs, spleen, peripheal blood, and bone marrow. \cite{cll-burger-med, cll-rozman-med}  There is no known cause for this disease, but it is suspected to have a genetic component. Loss or addition of large chromosomal material followed by additional mutations, that render the leukemia increasingly aggressive, are often observed. Additionally, mutations in \textit{IGHV} (immunoglobulin heavy variable) genes distinguishing different types of clinical behaviours of CLL and are prognostic of patent outcome \cite{immunogl-med}. \par
%Treatment of Chronic Lymphocytic Leukemia - Burger Jan A.
%Chronic Lymphocytic Leukemia - Rozman et al
%Immunoglobulin heavy variable (IGHV) genes and alleles - Xochelli et al 
\vspace{0.4cm}
Different treatment strategies are available for patients suffering from CLL. It is important to notice that studies on early-stage disease were unable to show a benefit of early therapeutic interventions: treatment of patients with early stage CLL did not result in a survival benefit. For this reason, and not to fruitlessly trigger the development of drug resistance, patient in these stages should not be treated, but only monitored, until the disease becomes \textit{active}. The degree of \textit{activity} of CLL can be assessed using the following guidelines:
\begin{itemize}
    \item Progressive lymphocytosis with an increase of $\geq 50 \%$ over a 2 month period, or lymphocyte doubling time of less than 6 months.
    \item Worsening of anemia and/or thrombocytopenia
    \item Massive nodes (i.e., $\geq 10$ cm in longest diameter)
    \item Massive or symptomatic splenomegaly and hepatomegaly
    \item Autoimmune complications
    \item Functional extranodal involvement (e.g., skin, kidney, lung, spine)
    \item Significant weight loss and fatigue, fevers above 38 degrees for more than two weeks, night sweats. 
\end{itemize}
When the treatment becomes necessary, specialists can choose between different classes of drugs, targeting different aspects of the cellular structure (e.g. surface antibodies), metabolism and external microenvironment:
\begin{itemize}
\item Cytostatic agents: having the goal to stop cellular proliferation by interfering with the replication process. Examples are purine analogs, such as Cytarabine (Cyt), one of the two drugs that we will discuss in this presentation.
\item Monoclonal Antibodies: specifically built to interact with surface antigens that have been documented to be overexpressed in malignancies, like CD-20 and CD-52 receptors in B-cells blood cancers. They have the goal of guiding the immune system towards the cancerous cells. Examples are Rituximab and Alemtuzumab.
\item Signalling - targeting agents: having the goal of interfering with the embedded signalling pathways in order to trigger apoptosis. Ibrutinib (Ibr), the second drug discussed in this presentation, is a Bruton Tyrosine Kinase inhibitor that hinders the downstream propagation of the signal generated by bounded BCR receptors, stopping the activation of B cells survival pathways. 
\item Immunotherapy: different CART approaches are being explored.


