\subsection{Experimental setup}
To measure the growth rate of mCherry A20 cells, 20 mice were inoculated via the tail vein with $5 \times 10^4$ logarithmic phase cells in PBS. A20 cells begin to proliferate and appear in blood in 2-3 weeks after inoculation.
On Days 16 and 22 after inoculation and prior to initiation of drug therapy, blood was taken from each mouse, and mCherry A20 cells were measured. Treatments were derived by published protocols in which Cyt and Ibr had been used \textit{in vivo}. The 20 inoculated mice were divided into 5 groups: 
\begin{enumerate}
	\item Control group, which only received PBS,
	\item Cyt Low group, which received $0.12$ mg/kg of Cyt for 5 days,
	\item Cyt High group, which received $62.5$ mg/kg of Cyt for 3 days, 
	\item Ibr Low group, wich recieved $9$ mg/kg of Ibr in days 1-5 and 8-10, and
	\item Ibr High group, which received $18$ mg/kg of Ibr on days 1–5 and 8–10. 
\end{enumerate}

Blood was collected from all mice on Day 12 after the beginning of treatments and the frequency of A20 mCherry cells was measured using flow cytometry. The difference in frequency of these cells in the blood before and after treatment was used to calculate the leukemia growth index of each group. The difference in growth index between treated and non-treated mice was used to calculate the inhibition of cell growth as a function of treatment.
Data from the \textit{in vivo} experiments and parameters from Table 1 of paper were used to validate the model. Computer simulations were performed using fourth-order adaptive step Runge–Kutta integration, as implemented in the ODE45 subroutine of MATLAB.
