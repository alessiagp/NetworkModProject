The goal of the project was to perform a critical examination of the selected publication, focusing in particular on the computational elements. To build a context for the reader, some background information on CLL was provided, together with an outline of modern treatment strategies. More detailed information regarding the two drugs discussed in the paper, Cytarabine and Ibrutinib, were also supplied. \\
The mathematical model was described in all of its terms, and a stability analysis was performed to identify possible equilibrium points and classify their stability. Two biologically feasible equilibria were found: an unstable tumour-free equilibrium $Eq_{0}^{*}$ and an asymptotically stable equilibrium $Eq_{0}^{*}$, in which the number of cancer cells is at carrying capacity. In this condition, if the treatment was stopped and the residual diseases totalled to just a single malignant cell, cancer would still grow back to carrying capacity. This may represent a state of cancer cell dormancy, an adaptive strategy used by cancer cells to overcome drug cytotoxicity. This stage may persist until the complete metabolism of the drug, which would allow tumour growth to recur. \\
The treatment protocols were introduced, together with the experimental parameter estimation employed by the authors of the paper to derive the instantaneous growth rate $r$ and the cytotoxicity rate $\mu_{AC}$. One simulation regarding combined therapy was discussed. \\
Simulations were run to try to reproduce all the protocols listed in the paper. A discrepancy of three orders of magnitude between the dosage reported in the paper and the ones expected from the computations was observed, and it was decided to proceed by using the latter. Despite this, the percentage growth inhibition derived from running the different simulations with the correct protocols were matching the results of the paper in all but one case, that is Cyt High. This discrepancy is probably due to some inconsistency in the paper. An additional discrepancy was observed when trying to reproduce the results of figure A2 in the paper. \\ 
It was derived from the literature that the right administration strategy for Cytarabine is a constant infusion, as opposed to the injection strategy modelled by the third differential equation. Two new mathematical models were built to simulate the first condition, and simulations were run with the treatment protocol derived from paper \cite{cyt-3}. \\
Different additional criticalities were detected: the growth rate $r$ of A20 cells was estimated in immuno-competent Balb/c mice. Therefore, $R$ already carries, implicitly, the contribution of the immune system towards the reduction of the malignant mass. Yet, in the first equation of the mathematical model, there are explicit terms that reproduce the competition with the immunity effector cells, whose dynamics are modelled by the second equation. This redundancy can lead to a dangerous overestimation of the capabilities of the immune system to fight cancerous cells. It was also noticed that the model does not take into account the natural production of effector cells. For this reason, additional simulations were run, implementing an $e_{0}$ term to represent this behaviour. Finally, it was noticed that many parameters used for the drugs and drugs-cancer interactions were taken from a study (referenced with the number 36 in the original paper) that is not concerned with Cytarabine or Ibrutinib, but with Cyclophosphamide. \\
From the growth inhibitions percentages obtained, it was observed that Cyt, a drug not currently used to treat CLL, is more cytotoxic to A20 cells \textit{in vitro} than Ibr, which suggests a repurposing of this cancer drug. A numerical simulation of the potential effect of Cyt plus Ibr on A20 cells, performed in the original paper, predicted that such a combination could increase cytotoxicity and inhibit cancer cell growth by up to 95\%. It would now be valuable to test this combined treatment \textit{in vivo}, especially as these drugs have different modes of action. 
