Cyt is a drug not currently used to treat CLL that is more cytotoxic to A20 cells \textit{in vitro} than Ibr, which suggests a repurposing of this cancer drug. Indeed, this prediction was vindicated when tested in the animal model presented in this paper. 
A numerical simulation of the potential effect of Cyt plus Ibr on A20 cells predicted that such a combination could increase cytotoxicity and inhibit cancer cell growth by up to 95\%. It would now be valuable to test this combined treatment \textit{in vivo}, especially as these drugs have different modes of action. 
This model exhibits several stable states that depend on biologically related parameters: Analysis of stability shows that the free-tumor $Eqm^*_0$ equilibrium is not stable, which means that if there are no more cancer cells and the treatment is stopped, the model is in equilibrium without growth, albeit unstable. This may also represent a state of cancer cell dormancy, an adaptive strategy used by cancer cells to overcome drug cytotoxicity. This stage may persist until complete metabolism of the drug, which would allow tumor growth to recur. 
The fixed point $Eqm^*_1$ is a stable equilibrium reached when the number of cancer cells reaches its maximum. The system is not stable at $Eqm^*_2$ equilibrium with periodical chemotherapy, which is obtained when treatment is stopped before the cancer cells are completely removed.
The step of calculating the growth rate of cancer cells \textit{in vivo} has to be carried out for each type of cancer cell, and it's easier to perform for blood-borne cancers. Current models do not easily help personalized chemotherapy dosing, partially because tumor cell growth rates vary between patients. The validation of experimental model with simulations studies can aid in selecting an optimal range of dosages to test.
Furthermore, the model can be used to simulate combination drug therapy: the model predicted that a combination of Cyt and Ibr would lead to about 95\% killing of A20 cells. Such high rates of killing are not expected in clinical practice, mainly due to subsequent toxicities. This model allows to predict a potentially effective new combination of drugs. Further experiments in vivo may reveal that two drugs with different modes of action may have acceptable efficacy at a lower dosage. 
