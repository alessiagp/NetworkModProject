The deterministic simulations presented in the paper were replicated by using the MATLAB suite \texttt{ode45}. This library, dedicated to solve systems of differential equations in the form $y'=f(t,y)$, implemnts an adaptive-step Runge-Kutta integration algorithm of the fourth-order. 
We developed the following script to solve the system of equation (1), (2), (3) for each treatment condition.

\begin{lstlisting}

Tf = 1440; % Final time
tspan = [0;Tf];
X0 = [500,2500,0]; % Starting conditions [A(0),E(0),C(0)]
op1 = odeset('MaxStep',1,'RelTol',1e-2,'AbsTol',1e-4);

[t,X] = ode45(@model,tspan,X0,op1);

function dX = model(t,X)

%%% Parameters definition
r = 0.01;   % A20 growth rate [h^-1]
K = 4e6;    % Max A20 number [cells/mouse]
mu_ac = 0.012; % Cyt 62.5 mg/kg for 3 days cytotoxicity rate [h^−1]
mu_ca = mu_ac * 10; % Deactivation of drug due to killing of A20 [h^-1]
d = 1.25 * 2.4e18; % Cyt 62.5 mg/kg dose [molecules/mouse]
mu_c = 0.231; % Cyt chemical deactivation rate [h^-1]
mu_a = 2e-12; % Effectors-A20 interaction coefficient [h^-1]
a = 2e3;    % Drug amount producing 50% of max effect on A20 [molecules]
p = 4e-14; % Production rate of effectors stimulated by A20 [h^-1]
mu_ec = 417; % Mortality rate of drug on effector cells [h^-1]
c = 1e2;    % Num of A20 producing 50% of max immune activation [cells]
b = 5e6;    % Drug amount producing 50% max effect on healty cells [molecules]
mu_ea = 4e-15; % A20-effectors interaction coefficient [h^-1]
mu_e = 4e-5; % Death rate of effecors [h^-1]
tau = 24;

A = r*X(1)*(1-(X(1)/K)) - mu_a*X(1)*X(2) - ((mu_ac*X(1)*X(3))/(a+X(3)));
E = -mu_e*X(2) + (p*X(1)*X(2))/(c+X(1)) - mu_ea*X(1)*X(2)  - (mu_ec*X(2)*X(3))/(b+X(3));
if t(1) > 336 && rem(round(t(1),0),tau) == 0 && t(1) < 384
    C = d - mu_c*X(3) - (mu_ca*X(3)*X(1))/(a+X(1)); 
else
    C = - mu_c*X(3) - (mu_ca*X(3)*X(1))/(a+X(1));
end
dX = [A;E;C];
end

\end{lstlisting}

The code above is referred to the cytamidin high group. To simulate the other treatments it is sufficient to change the $\mu_{AC}$ and $d$ parameters and the times in the conditional statements.
To reproduce the behaviour of the shifted dirac-delta function found in equation (3) we used a conditional statement that checks, in between specific intervals, if the actual time is a multiple of $\tau$ (i.e., the time between two shots). If this condition holds, an amount of drug $d$ is added to the system. Otherwise, the number of drug molecules is allowed to decrease. 
The parameters for the equation were taken from the reference paper, whereas the parameters for the integration algorithm were adjusted in order to deal with the simulation of the treatments. In particular, the maximum step allowed was set to 1 hour to prevent the integrator to "skip" some treatments, and the tolerance was increased to avoid errors due to the abrupt increase in the number of drug molecules in correspondence to the delivery of treatments.
Since the number of molecules per dose $d$ used in the paper does not match the correct one calculated for the different conditions, we performed simulations both with the reported value $d \sim 10^{15}$ and the correct value $d \sim 10^{18}$. Moreover, we run additional simulations adding to equation (2) a term $e_0$ accounting for the natural production of effector cells. This parameter was derived form ref [36] of the main paper.
Given that in reality Cyt has to be delivered though infusion, we adapted our code to support a continuous intake of drug and performed some simulations by adapting the treatment protocols reported in [cytamidine paper].
From the results of the various simulations we calculated the percentage growth inhibition with respect to the control group two days after the end of the treatment (growth inhibition $= 1 - \frac{A_t}{A_{nt}}$, where $A_t$ is the number of cancer cells in presence of treatment two days after the end of the treatment and $A_{nt}$ is the number of cancer cells in the control at the same time.)
