The deterministic simulations presented in the paper were replicated by using the MATLAB suite \texttt{ode45}. This library, dedicated to solve systems of differential equations in the form $y'=f(t,y)$, implemnts an adaptive-step Runge-Kutta integration algorithm of the fourth-order. 
A MATLAB script to solve the system of equation (1), (2), (3) for each treatment condition was developed and reported in a GitHub repository dedicated to this project \cite{project-repo}.

The \texttt{simulations} folder contains the scripts used to run the simulation of our ODE system and the scripts used to plot the results. The folder \texttt{sim\_data} contains the simulation results in form of CSV files, whose column represent: time, number of cancer cells, number of effector cells, number of drug molecules. The file names indicate the conditions, i.e.
\begin{itemize}
	\item No treatment
	\item Cyt\_high: Cyt $62.5 mg/kg$ 3 days
	\item Cyt\_low: Cyt $0.12 mg/kg$ 5 days
	\item Ibr\_high: Ibr $9 mg/kg$ days 1-5 and 8-10
	\item Ibr\_low: Ibr $18 mg/kg$ days 1-5 and 8-10
	\item Cyt\_Ibr: Cyt $62.5 +$ Ibr $9 mg/kg$ days 1-5 and 8-10
\end{itemize}
These were taken directly from the paper. The folder \texttt{figures} contains the plots of the CSV files obtained using the script \texttt{plotter.py} found in the main folder.\\

\subsection{Scripts}

\subsubsection{\texttt{model\_no\_trt.m}}
This script contains a simplified model without treatment, simulated with initial conditions $A = 1000,E = 3.5 \cdot 10^{5}$. As expected given the instability of the tumor - free equilibrium, the number of A20 malignant cells rapidly grows to carrying capacity.

\subsubsection{\texttt{model\_trt.m}}
The script at issue contains the final model of ODEs, and has been used to reproduce the 5 treatment protocols tested by the authors of the paper. The options for the numerical solution of the model are: \texttt{'MaxStep'=1} to avoid that the integration skips a treatment. \texttt{'RelTol' = 1e-2} and \texttt{'AbsTol' = 1e-4} to avoid numerical errors due to the abrupt change in the number of drug molecules when a treatment is performed.\\
To reproduce the behaviour of the shifted dirac-delta function found in equation (3) we used a conditional statement that checks, in between specific intervals, if the actual time is a multiple of $\tau$ (i.e., the time between two shots). If this condition holds, an amount of drug $d$ is added to the system. Otherwise, the number of drug molecules is allowed to decrease.\\
The parameters for the equation were taken from the reference paper, whereas the parameters for the integration algorithm were adjusted in order to deal with the simulation of the treatments. In particular, the maximum step allowed was set to 1 hour to prevent the integrator to \textit{skip} some treatments, and the tolerance was increased to avoid errors due to the abrupt increase in the number of drug molecules in correspondence to the delivery of treatments.\\
As it is, the script simulates the behaviour of the system in the presence of the mixed treatment Cyt_Ibr. However, to simulate the other protocols we just need to change the value of the parameter $\mu_{AC}$ and $d$, and additionally to modify the times in the conditional statements, in order to reflect the characteristics and requirements of each particualr treatment. \\
When reproducing the computations concerning the number of molecules per dose $d$ for the different treatment protocols, we noticed that they did not match the results obtained by the authors of the paper: in particualr, the values reported in the paper were a thousand times smaller than the correct results. We decided to perform simulations both with the reported value $d \sim 10^{15}$ and the correct value $d \sim 10^{18}$. The plots obtained using the latter are reported below, for each treatment protocol, while the plots obtained with $d \sim 10^{15}$ can be examined in the \texttt{figures} folder, and have been labelled with the suffix \textit{_1e15drug}. \\
Additionally, we noticed that the mathematical model does not take into account the natural production of effector cells: the second equation of the system admits a positive term only in the context of the recruitment effect carried out by the malignant cells on the immune population, without considering any other positive contributions. For this reason, we run additional simulations adding to equation (2) a term $e_0$ accounting for the natural production of effector cells. This parameter was derived form ref [36] of the main paper. The results of this set of simulations can be examined in the \texttt{figures} folder. \\
From the results of the various simulations we calculated the percentage growth inhibition of each group compared to the control group two days after the end of the treatment as:

$$ G.I. = 1 - \frac{A_t}{A_{nt}} $$ 

Where $A_t$ is the number of cancer cells in the treated group 28 hours after the end of the treatment and $A_{nt}$ is the number of cancer cells in the control at the same time. The values we obtained are reported in Table \ref{tab:gi}, alongside the values reported in the main paper. \\

\subsubsection{\texttt{model\_cyt\_infusion.m}}
The third equation of the mathemtical model introduced in the previous sections assumes that, independently from the drug, each treatment is administered by $\tau$ - interposed injections. However, as discussed in section 2.1.2, Cytarabine has to be delivered though a continuous infusion. We thus decided to build a new mathematical model, able to describe a constant intake of Cytarabine, by adapting the treatment protocols reported in \cite{cyt-3}. The modified system of equations can be appreciated in the files \texttt{model\_cyt\_infusion_high.m} and \texttt{model\_cyt\_infusion_low.m}. The protocols are as follows:
\begin{itemize}
	\item Cyt\_infusion\_low: Cyt 200 mg $\cdot$ 24h / m$^2$ for 7 consecutive days + 1000 mg $\cdot$ 3h / m$^2$ every 12 h for 6 days.
	\item Cyt\_infusion\_high: Cyt 1000 mg $\cdot$ 3h / m$^2$ every 12 h for 5 days + 2000 mg $\cdot$ 6h / m$^2$ every 12 h on days 12, 13, 15, 17.
\end{itemize}
To simulate these therapeutic regimes, we had to convert the doses from $[mg / m^2]$ of human surface area in $[molecules /mouse]$ according to standard guidelines [10.4103/0976-0105.177703] and to calculate the corresponding cytotoxicity rate $\mu_{AC}$. The latter calculation was done using the line 

$$ \mu_{AC} = 0.00018 \cdot \small{dose} [mg \cdot h/kg]+ 0.00098 $$

Interpolating the two data points present in \cite{main-paper}. 
Then, we needed the number of molecules that enter the body at each time-step during the infusion, to estimate this we used the mean value of the time-step when the treatment are delivered, calculated by looking at the vectors of time produced in of some preliminary simulations. The results of these simulations can be examined in the \texttt{figures} folder, and have been labelled with the prefix \texttt{Cyt_infusion}. 
