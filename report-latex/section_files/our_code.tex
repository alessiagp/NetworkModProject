The deterministic simulations presented in the paper were replicated by using the MATLAB suite \texttt{ode45}. This library, dedicated to solving systems of differential equations in the form $y'=f(t,y)$, implements an adaptive-step Runge-Kutta integration algorithm of the fourth order. 
A MATLAB script to solve the system of equations (1), (2), and (3) for each treatment condition was developed and reported in a GitHub repository dedicated to this project \cite{project-repo}.

The \texttt{simulations} folder contains the scripts used to run the simulation of the ODE system and the scripts used to plot the results. The folder \texttt{sim\_data} contains the simulation results in form of CSV files, whose columns represent: time, number of cancer cells, number of effector cells, and number of drug molecules. The file names indicate the conditions, i.e.
\begin{itemize}
	\item no\_trt: No treatment
	\item Cyt\_high: Cyt $62.5 mg/kg$ 3 days
	\item Cyt\_low: Cyt $0.12 mg/kg$ 5 days
	\item Ibr\_high: Ibr $9 mg/kg$ days 1-5 and 8-10
	\item Ibr\_low: Ibr $18 mg/kg$ days 1-5 and 8-10
	\item Cyt\_Ibr: Cyt $62.5 +$ Ibr $9 mg/kg$ days 1-5 and 8-10
\end{itemize}
These were taken directly from the paper. The folder \texttt{figures} contains the plots of the CSV files obtained using the script \texttt{plotter.py} found in the main folder.\\

\subsection{Scripts}

\subsubsection{\texttt{model\_no\_trt.m}}
This script contains a simplified model without treatment, simulated with initial conditions $A = 1000, E = 3.5 \cdot 10^{5}$. As expected given the instability of the tumour-free equilibrium, the number of A20 malignant cells rapidly grows to carry capacity.

\subsubsection{\texttt{model\_trt.m}}
This script contains the final model of ODEs and has been used to reproduce the 5 treatment protocols tested by the authors of the paper. \\
To reproduce the behaviour of the shifted Dirac-delta function found in equation (3) a conditional statement was implemented, that checks, in between specific intervals, if the actual time is a multiple of $\tau$ (i.e., the time between two shots). If this condition holds, an amount of drug $d$ is added to the system. Otherwise, the number of drug molecules is allowed to decrease.\\
The parameters for the equation were taken from the reference paper, whereas the parameters for the integration algorithm were adjusted to deal with the simulation of the treatments. In particular, the maximum step allowed was set to 1 hour to prevent the integrator from \textit{skip} some treatments, and the tolerance was increased to avoid errors due to the abrupt increase in the number of drug molecules in correspondence to the delivery of treatments.\\
The only requirement to simulate the different treatments protocols is to change the value of the parameter $\mu_{AC}$ and $d$, and modify the times in the conditional statements, to reflect the characteristics and requirements of each treatment. \\
When reproducing the computations concerning the number of molecules per dose $d$ for the different treatment protocols, it was evident that these did not match the results obtained by the authors of the paper: in particular, the values reported in the paper were a thousand times smaller than the correct results. For this reason, it was decided to perform simulations both with the reported value $d \sim 10^{15}$ and the correct value $d \sim 10^{18}$. The plots obtained using the latter are reported below, for each treatment protocol, while the plots obtained with $d \sim 10^{15}$ can be found in the \texttt{figures} folder, and have been labelled with the suffix \texttt{\_1e15drug}. \\
It was additionally noticed that the mathematical model does not take into account the natural production of effector cells: the second equation of the system admits a positive term only in the context of the recruitment effect carried out by the malignant cells on the immune population, without considering any other positive contributions. For this reason additional simulations were run, with a modified model that included, in the second differential equation, a term $e_0$ accounting for the natural production of effector cells. This parameter was derived from reference number 36 of the main paper. The results of this set of simulations can be examined in the \texttt{figures} folder, and have been labelled with the suffix \texttt{\_e0}.\\
The percentage growth inhibition, computed for each group by comparing the latter with the control group two days after the end of the treatment, was recovered from the results of the various simulation by exploiting the identity:

$G.I. = 1 - \frac{A_t}{A_{nt}}$ 

Where $A_t$ is the number of cancer cells in the treated group 28 hours after the end of the treatment and $A_{nt}$ is the number of cancer cells in the control at the same time. The values obtained are reported in Table \ref{tab:gi}, alongside the values reported in the main paper.

\subsubsection{\texttt{model\_cyt\_infusion.m}}
The third equation of the mathematical model introduced in the previous sections assumes that, independently from the drug, each treatment is administered by $\tau$ - interposed injections. However, as discussed in section 2.1.2, Cytarabine has to be delivered through a continuous infusion. 
It was thus decided to build a new mathematical model, able to describe a constant intake of Cytarabine, by adapting the treatment protocols reported in \cite{cyt-3}. The modified system of equations can be appreciated in the files \texttt{model\_cyt\_infusion\_high.m} and \texttt{model\_cyt\_infusion\_low.m}. The protocols are as follows:

\begin{itemize}
	\item \texttt{Cyt\_infusion\_low}: Cyt 200 mg $\cdot$ 24h / m$^2$ for 7 consecutive days + 1000 mg $\cdot$ 3h / m$^2$ every 12 h for 6 days.
	\item \texttt{Cyt\_infusion\_high}: Cyt 1000 mg $\cdot$ 3h / m$^2$ every 12 h for 5 days + 2000 mg $\cdot$ 6h / m$^2$ every 12 h on days 12, 13, 15, 17.
\end{itemize}

To simulate these therapeutic regimes, it was necessary to convert the doses from $[mg / m^2]$ of human surface area in $[molecules$ $/mouse]$ according to standard guidelines \cite{dose-conversion} and to calculate the corresponding cytotoxicity rate $\mu_{AC}$. The latter calculation was done using the line: 

$$ \mu_{AC} = 0.00018 \cdot \small{dose} [mg \cdot h/kg]+ 0.00098 $$

Interpolating the two data points present in \cite{main-paper}. 

Then, the number of molecules that enter the body at each time-step during the infusion was needed. Estimating this required the mean value of the time-step when the treatment is delivered, calculated by looking at the vectors of time produced in some preliminary simulations. The results of these simulations can be examined in the \texttt{figures} folder, and have been labelled with the prefix \texttt{Cyt\_infusion}. 
