The \textbf{growth rate \textit{in vivo} $r$ of A20 mCherry cells} was calculated according to

\[ r = \frac{\ln{N(t)/N(0)}}{t} = \frac{\ln{16338/3662}}{144} = 0.01\ h^{-1} \]

Where $N(0), N(t)$ are the number of cells at times $0$ (16 days after inoculation) and $t$ (22 days after inoculation).\\
To determine the \textbf{efficacy of Cyt and Ibr \textit{in vivo}}, five groups of mice were treated with different doses and time periods. Tail vein blood was collected on the day of initiation of treatment and two days after the last treatment. 
At each time point, the percent change in frequency of A20 cells in each treated mouse relative to the average frequency in the Control group was calculated. From these data, the average A20 frequency change for each treated group was obtained. 
The difference between the average frequency in each test group and the Control group represents the percent growth inhibition as a function of treatment. Growth inhibition due to Cyt was dose dependent (low dose 9\%, high dose 58\%), whereas inhibition due to Ibr was not (low and high doses about 44\%).

