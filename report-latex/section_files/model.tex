Based on previous studies, an ODE model was formulated to explain the interaction between CLL cells, immune cells and chemotherapeutic drugs:

\[
\begin{cases} 
	\frac{dA}{dt} = rA \bigl( 1 - \frac{A}{K} \bigr) - \mu_A AE - \frac{\mu_{AC} AC}{a+C} & (1)\\ \\
	\frac{dE}{dt} = -\mu_E E + \frac{pAE}{c+A} - \mu_{AE} AE - \frac{\mu_{EC} EC}{b+C} & (2) \\ \\
	\frac{dC}{dt} = \sum_{m=0}^{N-1} d\delta (t-m\tau) - \mu_{C} - \frac{\mu_{CA} CA}{a+A} & (3) 
\end{cases}
\]

$\frac{dA}{dt}$ describes the dynamic of A20 mCherry cells. The first term reflects the assumption that cancer cells follow a logistic growth with \textit{instantaneous growth rate} $r$ and carrying capacity $K$. The carrying capacity represents the maximal tumour cell number that the system can host. Logistic growth is a reasonable assumption for cancer growth since it takes place in a competitive environment with limited resources. Cancer cells can be killed by both NK cells and T CD8 lymphocytes: these were considered together in the single variable $E$, whose dynamic is described in the second equation. The overall killing activity of immune cells can be modelled with the law of mass action, assuming a \textit{killing efficiency} $\mu_{A}$. The last term represents the effect of the treatment on tumour cells: the numerator simulates the interaction between tumour cells and drug molecules with the law of mass action, with a \textit{killing efficiency} $\mu_{AC}$, while the denominator introduces a Michaelis Menten drug saturation response, for which the whole term converges to the maximum killing rate $\mu_{AC} \cdot A$ as the drug concentration $C$ is brought to infinity. This is a reasonable strategy to model the drug response since a plateau in the effectiveness of the drug is expected, as its concentration is increased. In this last term, $a$ represents the drug concentration needed to reach half of the maximum killing rate. \par
\vspace{0.4cm}
$\frac{dE}{dt}$ describes the dynamic of immune effector cells. Their number is assumed to decline with rate $\mu_{E}$ due to natural death. It is known from the literature that cancer cells can induce a recruitment effect on immune cells, due to the pro-inflammatory environment defined by cancer itself. This is represented by the second term. The recruitment effect increases as the tumour mass grows, but up to a certain maximum rate, represented by $p \cdot E$. Additionally, $c$ is the number of cancer cells by which the immune system response is half of its maximum. It is also known from the literature that T CD8 and NK cells undergo apoptosis after a certain number of encounters with malignant cells: the cytotoxic molecules released against cancer inevitably cause damage to immune cells too, and this is modelled by the third term, using again the law of mass action. Finally, the drug administered to treat CLL also kills host immune cells: this is modelled as described above for cancer cells. \par
\vspace{0.4cm}
$\frac{dC}{dt}$ describes the first-order pharmacokinetics of a drug with an external source, with $C$ being the concentration of the drug in the bloodstream. A dose $d$ of the drug is injected every $\tau$ hours. By modelling the injection as a shifted Dirac Delta function $\delta (t - m\tau)$, the $m^{th}$ dose raises $C(t)$ by $d$ units at $t=m\tau$. It was assumed that the drug was eliminated from the body with a rate $\mu_C$, calculated as $\mu_C = \frac{\ln 2}{t_{1/2}}$, where $t_{1/2}$ is the elimination half-life of the drug (1–3 h for Cyt and 4–6 h for Ibr). The drug concentration can also be depleted by the interaction with cancer cells, having rate $\mu_{CA}$. Finally, $a$ represents the drug concentration producing $50\%$ of the maximum activity in the A20 mCherry cell population. 
