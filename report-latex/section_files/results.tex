A tentative to reproduce the results obtained by the paper was done.
From the results of the various simulations, the percentage growth inhibition was calculated with respect to the control group two days after the end of the treatment as
$$ G.I. = 1 - \frac{A_t}{A_{nt}} $$ 
Where $A_t$ is the number of cancer cells in the treated group 28 hours after the end of the treatment and $A_{nt}$ is the number of cancer cells in the control at the same time. The values we obtained are reportes in Table \ref{tab:gi}, alongside the values reported in the main pater.

\begin{table}[]
\begin{tabular}{|c|c|l|l|}
\hline
	Group & \begin{tabular}[c]{@{}c@{}}Growth \\ Inhibition (\%)\end{tabular} & GI (paper) & GI (exp) \\ \hline
Cyt low           & 11 & 10.0 & 9.0    \\ \hline
Cyt high          & 70 & 59.0 & 58.0   \\ \hline
Ibr low           & 44 & 43.4 & 43.5 \\ \hline
Ibr high          & 45 & 44.0 & 44.5 \\ \hline
Cyt + Ibr         & 96 & 95.0 & -    \\ \hline
Cyt infusion low  & 93 & -    & -    \\ \hline
Cyt infusion high & 98 & -    & -    \\ \hline
\end{tabular}
\caption{Growth inhibition calculated 2 days after the end of the treatments, with respect to the control group in the same day}
\label{tab:gi}
\end{table}

The discrepancy in the growth inhibition for the Cyt high group, that was calculated at the time-point corresponding to two days after the end of the treatments, is probably due to some inconsistencies in the main paper, in which it is not clear when such value was calculated.\\
The results of the simulations without treatments are identical to the ones reported (Figure A1 in the paper), and so is the growth inhibition due to the combined use of a low dose of Ibr and a high dose of Cyt for 8 days (figure 6 in the paper). However, there is a mismatch in what was obtained from this project in trying to reporuce figure A2 and the figure itself. 
In particular, the most striking difference is that the tumor-free period we obtained after the 30 days of combined treatment is much shorter.\\
Regarding the adaptation of a treatment protocol involving the delivery of Cyt through infusion, we took as reference the paper \cite{cyt-3}, in which Cyt is used to treat acute myeloid leukemia in humans. Doses where adjusted according to standard guidelines \cite{dose-conversion} in order to pass from humans to mice. 
It has to be noted that these treatments involved the combined use of Cyt and another drug, which parameters have not been found. Therefore, the simulations are restricted to a therapy regiment with Cyt only.
